{\fontsize{9bp}{1em}\selectfont % should be 9pt
\noindent\begin{tabular}{>{\raggedright}p{.25\textwidth}p{.7\textwidth}}
  \multicolumn{2}{l}{\textbf{Beneficiary ViCOROB research institute (Universitat de Girona)}} \\\midrule
\textbf{General Description} &
ViCOROB research institute belongs to the Department d'Aquitectura i Technologia de Computadors at the Universitat de Girona, a public university in Girona since 1992. ViCOROB is a research institute specialized in computer vision and robotics at \ac{udg}. In 2013, the \ac{udg} has rewarded ViCOROB by promoting the group into a Research Institute funded by the university itself.
VICOROB has been always highly motivated to solve different and challenging  societal problems and succeeded to obtain outside funding for solving them. The scientific results have been disseminated not only in form of peer-reviewed articles but also to the broad public audience by participating in several media events, speeches and published material. Three spin-off companies emerged: Coronis Computing SL, AQSENSE and BonesNotes.
\\\midrule
\textbf{Role and Commitment of key persons (supervisor)} &
Dr. Robert Mart\'i, PhD, is associate professor in the Image Analysis Lab within ViCOROB. His main research interests are in the field of medical image analysis, specially focusing on feature extraction, pattern-recognition and image registration and its application to mammographic and prostate image analysis and \ac{cad} system.
\\\midrule
\textbf{Key Research Facilities, Infrastructure and Equipment} &
The laboratories of ViCOROB are well-equipped with computers, servers, and specific software required for processing clinically data acquired. The Image Analysis Lab has recently been equipped with 2 high-performance servers (featuring 4 quad-core processors and 128 GB of RAM), a Totoku MS31i2 Diagnostic Displays system, and access to the use of CESCA facilities (the Supercomputing research center in Barcelona) which offers supercomputing shared-memory and distributed-memory machines suitable when dealing with such huge amount of data.
\\\midrule
\textbf{Independent research premises?} &
Yes --- 2 clusters with 32 nuclei for massive and parallel computing
\\\midrule
\textbf{Previous Involvement in Research and Training Programmes} &
During the last 3 years, \ac{udg} has coordinated 6 individual MCA and 2 Research Networks (RESKITCHLAB and CHEMEVE). In the last decade, the \ac{udg} has participated in more than 160 European projects. The following most noticeable research projects related to medical imaging have been developed at ViCOROB: Proscan (Help with location of prostate cancer) and M3CAD (Multi-modality and Multi-view Mammographic Computer Aided Diagnosis System)
\\\midrule
\textbf{Current involvement in Research and Training Programmes} &
\ac{udg} is currently coordinating an ITN action, SANITAS, and is participating as a full partner in ENDURE and ROBOACADEMY. Also, \ac{udg} is coordinating 3 IRSES actions (CANIOC, CLIMSEAS, and IREBD) and participating in one IAPP (PEP2BRAIN). Moreover, it is the main beneficiary of 8 individual Marie Curie actions. \ac{udg} is coordinating two Starting Grant projects (ERC) one Proof of Concept (ERC) and one COST action, among other participation, both as a partner and coordinator, in R\&D European and national funded projects.
Current projects under development in ViCOROB include: ASSURE (Adapting Breast Cancer Screening Strategy Using Personalised Risk Estimation), IA-BioBreast (temporal analysis and automatic detection of lesions in multimodal images). Furthermore, ViCOROB organises the Erasmus Mundus Master in Computer Vision and Robotics (Vibot) and the Erasmus+ Joint Master in Medical Imaging and Applications (MaIA). 
\\\midrule
\textbf{Relevant Publications and/or research/innovation products} &
\begin{itemize}[noitemsep]
\item \textbf{R. Mart\'i et al.}, ``Computer-Aided Detection and Diagnosis for prostate cancer based on mono and multi-parametric MRI: A review'', \textit{Computers in Biology and Medicine}, vol. 60, pp 8 - 31, 2015). [IF 1.475, Q2(41/85) B] 
\item \textbf{R. Mart\'i et al.}, ``A supervised learning framework of statistical shape and probability priors for automatic prostate segmentation in ultrasound images'', \textit{Medical Image Analyis}, 7(6), pp 587-600, 2013. [IF 4.087, Q1(7/115) CSAI] 
\item \textbf{R. Mart\'i et al.}, ``A spline-based diffeomorphism for prostate multimodal registration'', \textit{Medical Image Analyis}, 16(6), pp 1259-1279. 2012. [IF 4.087, Q1(7/115) CSAI] 
\item \textbf{R. Mart\'i et al.}, ``A survey of prostate segmentation methodologies in ultrasound, magnetic resonance, and computed tomography images'', \textit{Computer Methods and Programs in Biomedicine}, 108(1), pp 262-287. 2012. [IF 1.555, Q1(21/100) CSTM]
\item \textbf{R. Mart\'i et al.}, ``Statistical shape and texture model of quadrature phase information for prostate segmentation'', \textit{International Journal of Computer Assisted Radiology and Surgery}, 7(1), pp 43-55, 2012. [IF 1.364, Q3(76/120) RNMMI]
\end{itemize}
\\\bottomrule
\end{tabular}}
\vspace{\baselineskip}

{\fontsize{9bp}{1em}\selectfont
\noindent\begin{tabular}{>{\raggedright}p{.25\textwidth}p{.7\textwidth}}
  \multicolumn{2}{l}{\textbf{Partner Organisation Florida State University}} \\\midrule
\textbf{General Description} &
Florida State University (FSU)
\\\midrule
\textbf{Key Persons and Expertise (supervisor)} &
Anke Meyer-Baese, PhD, Professor at the Department of Scientific Computing in the \ac{fsu}
\\\midrule
\textbf{Key Research facilities, infrastructure and equipment} &
National high magnetic field laboratory. \ac{fsu} research foundation, multi-parametric \ac{mri}, MR scanners. Clinical resources, clusters for intensive computing.
\\\midrule
\textbf{Previous and Current Involvement in Research and Training Programmes} &
Prof. Anke Meyer-Baese led over twenty funded research projects (NSF, NIH) in her field with a total volume of six million dollars.
Currently, she directs among others, a research project on \ac{cad} for breast cancer with NIH funding.
Her interaction with students is exemplary: she directed two research professors, six post-docs, over 21 PhD students graduated and 6 current PhD students, achieving teaching evaluations among the best at \ac{fsu}, and attaining one of the highest students' retentions. Many of her former doctoral and postdoctoral students have obtained positions in academia.
\\\midrule
\textbf{Relevant Publications and/or research/innovation product} &
\begin{itemize}[noitemsep]
\item \textbf{A. Meyer-Baese, V. J. Schmid}, ``Pattern Recognition and Signal Analysis in Medical Imaging'', \textit{Elsevier}
\item \textbf{A. Meyer-Baese et al.}, ``Global exponential stability of competitive neural networks with different time scales'', \textit{Neural Networks, IEEE Transactions on}, 14(3), pp 716-719
\item \textbf{A. Meyer-Baese et al.}, ``Comparison of two exploratory data analysis methods for fMRI: unsupervised clustering versus independent component analysis'', \textit{Information Technology in Biomedicine, IEEE Transactions on}, 9(3), pp 387-398
\end{itemize}
\\\bottomrule
\end{tabular}}

