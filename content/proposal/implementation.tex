\section{QUALITY AND EFFICIENCY OF THE IMPLEMENTATION}
\label{sec:implementation}

\subsection{Coherence and effectiveness of the work plan}

The proposal is split into 7 work packages (WPx):
\textbf{WP1}: Data acquisition and dissemination (duration of 5 months);
\textbf{WP2}: Pre-processing (duration of 7 months);
\textbf{WP3}: Segmentation (duration of 7 months);
\textbf{WP4}: Registration (duration of 7 months);
\textbf{WP5}: Detection and assessment (duration of 8 months);
\textbf{WP6}: \ac{pirads} evaluation (duration of 2 months);
\textbf{WP7}: Project management (duration of 36 months).

% \begin{itemize}[noitemsep]
% \item[] \textbf{WP1: Data acquisition and dissemination (duration of 5 months)} ---
%   % \begin{itemize}[noitemsep,nolistsep]
%   % \item[] \textbf{T1.1} Prepare the multi-parametric \ac{mri} dataset
%   % \item[] \textbf{T1.2} Finalise the web-platform
%   % \item[] \textbf{T1.3} Publicly publish the dataset
%   % \item[] \textbf{T1.4} Organise a \emph{Grand-challenge}
%   % \end{itemize}
% \item[] \textbf{WP2: Pre-processing (duration of 7 months)} ---
%   % \begin{itemize}[noitemsep,nolistsep]
%   % \item[] \textbf{T2.1} Evaluate the bias field correction methods on our public dataset (1.3)
%   % \item[] \textbf{T2.2} Evaluate the denoising methods on our public dataset (1.3)
%   % \item[] \textbf{T2.3} Extend the normalisation methods on our public dataset (1.3)
%   % \item[] \textbf{T2.4} Apply and evaluate \ac{mrsi} pre-processing methods on our public dataset (1.3)
%   % \end{itemize}
% \item[] \textbf{WP3: Segmentation (duration of 7 months)} ---
%   % \begin{itemize}[noitemsep,nolistsep]
%   % \item[] \textbf{T3.1} Design the hybrid segmentation method
%   % \item[] \textbf{T3.2} Develop the segmentation
%   % \item[] \textbf{T3.3} Validate the segmentation method on our public dataset (1.3)
%   % \end{itemize}
% \item[] \textbf{WP4: Registration (duration of 7 months)} ---
%   % \begin{itemize}[noitemsep,nolistsep]
%   % \item[] \textbf{T4.1} Extend the current registration method
%   % \item[] \textbf{T4.2} Validate the registration method on our public dataset (1.3)
%   % \end{itemize}
% \item[] \textbf{WP5: Detection and assessment (duration of 8 months)} ---
%   % \begin{itemize}[noitemsep,nolistsep]
%   % \item[] \textbf{T5.1} Design the classifier architecture
%   % \item[] \textbf{T5.2} Develop the classifier
%   % \item[] \textbf{T5.3} Validate the classifier on our public dataset (1.3)
%   % \end{itemize}
% \item[] \textbf{WP6: \ac{pirads} evaluation (duration of 2 months)} ---
% \item[] \textbf{WP7: Project management (duration of 36 months)}
%   % \begin{itemize}[noitemsep,nolistsep]
%   % \item[] \textbf{T6.1} Transpose the \ac{pirads} markers to computer vision marker
%   % \item[] \textbf{T6.2} Develop the \ac{pirads}-based grading
%   % \item[] \textbf{T6.3} Validate this approach on our public dataset (1.3)
%   % \end{itemize}
% % \item[] \textbf{WP7: Communication to general public (duration of 36 months)}
% %   \begin{itemize}[noitemsep,nolistsep]
% %   \item[] \textbf{T7.1} Internal seminars in FSU and UdG
% %   \item[] \textbf{T7.2} Publish in general public press such as \emph{Engega} 
% %   \item[] \textbf{T7.3} Scientific blog with periodic entries 
% %   \item[] \textbf{T7.4} Enrollment to educational workshop such as \emph{Yong Research Campus} at UdG
% %   \end{itemize}
% % \item[] \textbf{WP8: Scientific dissemination (duration of 36 months)}
% %   \begin{itemize}[noitemsep,nolistsep]
% %   \item[] \textbf{T8.1} Maintenance of the web-platform by publishing code and results
% %   \item[] \textbf{T8.2} Publish the works in peer-review journals
% %   \item[] \textbf{T8.3} Publish the works in peer-review international conferences
% %   \end{itemize}
% \end{itemize}

The following deliverables (Dx) will be released during the project:
% \begin{itemize}[noitemsep]
% \item[] \textbf{D1.1} Multi-parametric \ac{mri} database
% \item[] \textbf{D2.1} Software toolbox for multi-parametric \ac{mri} images pre-processing
% \item[] \textbf{D3.1} Software toolbox for multi-parametric \ac{mri} images registration
% \item[] \textbf{D4.1} Software toolbox for multi-parametric \ac{mri} images segmentation
% \item[] \textbf{D5.1} Software toolbox for multi-parametric \ac{mri} images classification
% \item[] \textbf{D6.1} Software toolbox for \ac{pirads} grading
% \end{itemize}
\textbf{D1.1} \ac{mpmri} database and report with dataset presentation and online availability of the dataset;
\textbf{D1.2} Submit to Grand-Challenge at MICCAI 2018, ``Prostate detection and grading using \ac{mpmri}'';
\textbf{D2.1} Submit to ISBI 2018, ``Empirical evaluation of bias field correction methods for \ac{mpmri}'' \& ``Empirical evaluation of noise reduction methods for \ac{mpmri}'';
\textbf{D2.2} Submit to MICCAI 2018, ``Normalisation techniques for \ac{mpmri}'';
\textbf{D2.3} Submit to IEEE TMI, ``Pre-processing tools for \ac{mpmri}'';
\textbf{D2.4} Toolbox for \ac{mpmri} images pre-processing;
\textbf{D3.1} Submit to PROMISE12 challenge of results using only \ac{t2w}-\ac{mri};
\textbf{D3.2} Submit to MedIA, ``Zonal segmentation of the prostate using deep-learning and \ac{mpmri}'';
\textbf{D3.3} Toolbox for \ac{mpmri} images segmentation;
\textbf{D4.1} Submit to SPIE Medical Imaging 2019, ``Prostate registration using \ac{mpmri} and spline-based non-linear diffeomorphism'';
\textbf{D4.2} Toolbox for \ac{mpmri} images registration;
\textbf{D5.1} Submit to IEEE TMI, ``Prostate cancer detection and assessment using deep-learning and \ac{mpmri}'';
\textbf{D5.2} Toolbox for \ac{mpmri} images classification;
\textbf{D6.1} Submit to JMRI, ``Evaluation of a prostate \ac{cad} systems using \ac{pirads}''.
\textbf{D6.2} Toolbox for \ac{pirads} grading.

To achieve the proposed goals, the following milestones (Mx) are defined:
% \begin{itemize}[noitemsep]
% \item[] \textbf{M1.1} Deadline December 2016 --- Grand-Challenge submission at the conference MICCAI 2017, ``Prostate detection and grading using multi-parametric \ac{mri}''
% \item[] \textbf{M2.1} Deadline October 2016 --- Paper submission at the conference ISBI 2017, ``Empirical evaluation of bias field correction methods for multi-parametric \ac{mri}''
% \item[] \textbf{M2.2} Deadline October 2016 --- Paper submission at the conference ISBI 2017, ``Empirical evaluation of noise reduction methods for multi-parametric \ac{mri}''
% \item[] \textbf{M2.3} Deadline March 2017 --- Paper submission at the conference MICCAI 2017, ``Normalisation techniques for multi-parametric \ac{mri}''
% \item[] \textbf{M2.4} Deadline March 2017 --- Paper submission to a peer-review journal, ``Pre-processing tools for multi-parametric \ac{mri}''
% \item[] \textbf{M3.1} Deadline October 2017 --- Submission to PROMISE12 challenge of results using only \ac{t2w}-\ac{mri}
% \item[] \textbf{M3.2} Deadline October 2017 --- Paper submission to peer-review journal \emph{Medical Image Analysis}, ``Zonal segmentation of the prostate using deep-learning and multi-parametric \ac{mri}''
% \item[] \textbf{M4.1} Deadline June 2018 --- Paper submission at the conference ICIP 2018, ``Prostate registration using multi-parametric \ac{mri} and spline-based non-linear diffeomorphism''
% \item[] \textbf{M5.1} Deadline January 2019 --- Paper submission to peer-review journal \emph{IEEE Transactions on Medical Imaging}, ``Prostate cancer detection and assessment using deep-learning and multi-parametric \ac{mri}''
% \item[] \textbf{M6.1} Deadline March 2019 --- Paper submission to peer-review journal \emph{Journal of Magnetic Resonance Imaging}, ``Evaluation of a prostate \ac{cad} systems using \ac{pirads}''
% \end{itemize}
\textbf{M1.1} Availability of \ac{mpmri} dataset;
\textbf{M1.2} Availability of ground-truth and anonymization/deidentification of the \ac{mpmri} dataset;
\textbf{M2.1} Implementation and results of pre-processing tools for \ac{mpmri};
\textbf{M3.1} Implementation of state-of-the-art segmentation methods;
\textbf{M3.2} Implementation and results of segmentation for \ac{mpmri} prostate;
\textbf{M4.1} Implementation of state-of-the-art registration methods;
\textbf{M4.2} Implementation and results of registration for \ac{mpmri} prostate;
\textbf{M5.1} Implementation of traditional features for classification of prostate cancer;
\textbf{M5.2} Implementation and results of \ac{cnn}-based classification for prostate cancer detection;
\textbf{M6.1} Completion of the framework.

\subsection{Appropriateness of the allocation of tasks and resources}

Each work package are subdivided into tasks (Tx):
% \textit{T1.1} Prepare the multi-parametric \ac{mri} dataset (3 months);
% \textit{T1.2} Finalise the web-platform (4 months);
% \textit{T1.3} Publicly publish the dataset (1 months);
% \textit{T1.4} Organise a \emph{Grand-challenge} (within 6 months).
% \textit{T2.1} Evaluate the bias field correction methods on our public dataset (1.3) (2 months);
% \textit{T2.2} Evaluate the denoising methods on our public dataset (1.3) (2 months);
% \textit{T2.3} Extend the normalisation methods on our public dataset (1.3) (2 months);
% \textit{T2.4} Apply and evaluate \ac{mrsi} pre-processing methods on our public dataset (1.3) (1 month).
% \textit{T3.1} Design the hybrid segmentation method (2 months);
% \textit{T3.2} Develop the segmentation (3 months);
% \textit{T3.3} Validate the segmentation method on our public dataset (1.3) (2 months).
% \textit{T4.1} Extend the current registration method (4 months);
% \textit{T4.2} Validate the registration method on our public dataset (1.3) (2 months).
% \textit{T5.1} Design the classifier architecture (2 months);
% \textit{T5.2} Develop the classifier (4 months);
% \textit{T5.3} Validate the classifier on our public dataset (1.3) (2 months).
% \textit{T6.1} Transpose the \ac{pirads} markers to computer vision marker (1 month);
% \textit{T6.2} Develop the \ac{pirads}-based grading (1 month);
% \textit{T6.3} Validate this approach on our public dataset (1.3) (1 month).
The WP1 for ``Data acquisition and dissemination'' is dedicated to work together with clinicians to collect the prostate mp-MRI and generate the associated ground-truth (T1.1 - 3 months).
In parallel, a web-platform will be developed to host source code and dataset (T1.2 - 4 months).
Additionally, the dataset will be anonymized and deidentified such that it can be proposed as Open Data on our web-platform (T1.3 - 1 months) and for a Grand-Challenge (T1.4 - within 6 months).
The WP2 for ``Pre-processing'' is dedicated to evaluate the bias correction algorithms on our public dataset (T2.1 - 2 months), denoising methods (T2.2 - 2 months), develop a normalization methods \ac{dw}-\ac{mri} modality (T2.3 - 2 months), and study pre-processing method for \ac{mrsi} (T2.4 - 1 months).
The WP3 for ``Segmentation'' is dedicated to the design (T3.1 - 2 months), development (T3.2 - 3 months), and evaluation (T3.3 - 2 months) of an hybrid segmentation method on the public dataset.
The WP4 for ``Registration'' is dedicated to extend the current registration method (T4.1 - 4 months) and validate this method on the public dataset (T4.2 - 2 months).
The WP5 for ``Detection and assessment'' is dedicated to the design (T5.1 - 2 months), development (T5.2 - 4 months), and evaluation (T5.3 - 2 months) of a deep-learning based prostate cancer detection system as well as is evaluation on the public dataset.
The WP6 for `` PI-RADS evaluation'' is dedicated to translate (T6.1 - 1 month) and develop (T6.2 - 1 month) the classification results to a PI-RADS evaluation.
The WP7 for `` Project management'' is dedicated the management of the project which will be a regular follow-up during the full duration of the fellowship, involving the experienced supervisors, clinicians, and the fellow.

\subsection{Appropriateness of the management structure and procedures, including quality management and risk management}

% As previously mentioned in the mentoring plan, a constant communication between the fellow and the two experienced supervisors will take place, on a weekly bases to monitor the advancement and assess the status of the specified milestones and the advancement of the project.
% Additionally, the development of the \ac{cad} solutions along the project --- usually associated with the milestones --- will be presented to clinicians to get feedback and therefore the right decisions for the next steps.
% Additionally, the \ac{oitt} will offer its support for administrative and financial management 

% This fellowship will provide the applicant with the generous and unique financial opportunity to accomplish the realization of the first prototype of \ac{cad} for prostate cancer detection.
% He is fully aware of the goal of this ambitious and interdisciplinary project.
% It is very important for him to determine more accurately whether or not the project is on schedule. 
% He will pay special attention to the most important conferences and special issues deadlines in the field (see Sect.\,\ref{sec:implementation} for further information). 
% They will be used as milestones to monitor progress and also as an opportunity to disseminate and share the results of the work with other cancer experts.
% A special budget will be allocated for them. 
% Increased attention will also be paid to the submission of international journal papers in journals with a high reputation in the field.

Three major tools are in place to ensure timely progress of the project.
First, \textbf{weekly meetings} with Prof. A. Meyer-Baese and/or Dr. R. Mart\'i will take place in which the fellow will present the main achievements of his work.
These meetings will be specially focused on addressing any difficulties found and working together to overcome them.
These meetings will provide unique opportunities to exchange ideas and develop new ones to complete the project.
The candidate will present a written deliverable to the supervisors at 5 (D1.1), 12 (D2.1-3), 18 (D3.2), 29 (D4.1), and 34 (D5.1) months outlining achieved objectives and planned research and training activities for the following period.
Second, a \textbf{project plan} has been established to easily monitor and evaluate the project against deliverables and milestones.
Finally, the \textbf{annual researchers evaluation} ensures that the project is developing smoothly.
The fellow has to complete a Contribution Review Form\footnote{https://www.sc.fsu.edu/forms/facultyperformance} with short- and long-term objectives.
The Executive Committee of the Department composed of professors in the fellow's area review this evaluation and ensure the project's objectives are met.
The combination of these three processes guarantees that the fellow will not face difficulties that could potentially block the project.

\ac{fsu} has a dedicated research support office for fellowships, \textbf{\ac{fsu} Research Foundation}, which oversees the financial management and administration of all fellowships grants and contracts stemming from non-federal money, and will ensure all financial and reporting arrangements are met.
They have an extensive experience in administrating grants of more than \$300 million dollars.
\ac{fsu} has participated in many FP7 projects and is hosting Marie Curie Fellowships.
The university has developed specific guidelines and processes on Marie Curie Actions to ensure that the contractual and reporting requirements are fully met.

As part of the \textbf{risk management}, tasks that may represent an issue have been identified and alternative plans established. 
The previous experience of the applicant in applying his developed machine learning approaches to research lines ranging from prostate, brain and skin cancer to retinal diseases, and the collective endeavor of all involved scientists and their research groups along with the intensive experience of the hosts institutions in managing projects, are indicative of the feasibility of this project. 
Nevertheless, G. Lema\^itre is well aware of the importance of developing a contingency plan to minimize the risks and present a mitigation strategy to surmount the possible issues:
% \begin{itemize}[noitemsep]
% \item \textbf{Risk 1(R1)} Potential problems: Insufficient number of cases. Probability: low. Alternative strategies:  Consult with major cancer centers in US and Spain. Contingency plan: Meeting and consulting with prostate cancer specialists at the Lee-Moffitt Center in Tampa, Florida.
% \item \textbf{Risk 2 (R2)} Potential problems: Insufficient pre-processing results. Probability: medium. Alternative strategies: Joint segmentation and registration implementation based on optical flow and active contour model. Contingency plan: Meeting and consulting with researcher and image registration expert Joachim Weickert, Saarland University (Germany).
% \item \textbf{Risk 3 (R3)} Potential problems: Non-representative feature extraction and poor classification results. Probability: Very low. Alternative strategies and contingency plan: apply dynamic texture techniques for spatio-temporal feature extraction and employ gray-level run-length matrix (GRRL) and gray level intensity size zone matrix (GLISZ). 
% This methodology has been a successful approach in different applications, and the applicant has the associated theoretical knowledge. 
% Consider selecting Random Forest Trees as alternative classifiers. Meeting and consulting with data mining expert, Claudia Plant, head of the scientific computing research group in the Helmholtz Zentrum in Munich.
% \item \textbf{Risk 4 (R4)} Description: Insufficient number of companies interested in the software. Probability: Low. Contingency plan: (i) Seek and arrange meetings with companies and the industry sector to show beta-version of the software at early stage. (ii) Schedule the meetings efficiently by not waiting until the last stage of the fellowship. (iii) Invite companies to monitor the development of the software. Thus, they could make recommendations in terms of interface and practical operation under an end user perspective. (iv) Request for support from OITT at the \ac{udg}, a service devoted to give advice on knowledge and technology transfer between university and industry.
% \end{itemize}
\textbf{Risk 1(R1)} Potential problems: Insufficient number of cases. Probability: low. Alternative strategies:  Consult with major cancer centers in US and Spain. Contingency plan: Meeting and consulting with prostate cancer specialists at the Lee-Moffitt Center in Tampa, Florida; 
\textbf{Risk 2 (R2)} Potential problems: Insufficient pre-processing results. Probability: medium. Alternative strategies: Joint segmentation and registration implementation based on optical flow and active contour model. Contingency plan: Meeting and consulting with researcher and image registration expert Joachim Weickert, Saarland University (Germany);
\textbf{Risk 3 (R3)} Potential problems: Non-representative feature extraction and poor classification results. Probability: Very low. Alternative strategies and contingency plan: apply dynamic texture techniques for spatio-temporal feature extraction and employ gray-level run-length matrix (GRRL) and gray level intensity size zone matrix (GLISZ). 
This methodology has been a successful approach in different applications, and the applicant has the associated theoretical knowledge. 
Consider selecting Random Forest Trees as alternative classifiers. Meeting and consulting with data mining expert, Claudia Plant, head of the scientific computing research group in the Helmholtz Zentrum in Munich;
\textbf{Risk 4 (R4)} Description: Insufficient number of companies interested in the software. Probability: low. Contingency plan: (i) Seek and arrange meetings with companies and the industry sector to show beta-version of the software at early stage. (ii) Schedule the meetings efficiently by not waiting until the last stage of the fellowship. (iii) Invite companies to monitor the development of the software. Thus, they could make recommendations in terms of interface and practical operation under an end user perspective. (iv) Request for support from \ac{oitt} at the \ac{udg}, a service devoted to give advice on knowledge and technology transfer between university and industry.

\subsection{Appropriateness of the institutional environment (infrastructure)}
\label{sec:institution}

\emph{Florida State University} 
According to Shaingai's ARWUA, \ac{fsu} is between the 200 best universities worldwide. 
Its research infrastructure includes one of the largest high magnetic field laboratory worldwide, closely related to this project. 
(i) Institutional resources: One 7 T, two 3 T and two high-end 1.5 T MR scanners for image acquisition,
(ii) Clinical resources: Excellent facilities and experience in scientific studies aiming at the evaluation of \ac{cad} methods for MR mammography, counting on the collaboration of Prof. Adrian Barbu, a computer scientist with expertise in \ac{cad} design. 
This aspect can be essential for documenting the real-world impact of research progress in the project w.r.t. practical health care. 
(iii) Workstation facilities: office space equipped with a desktop computer and network connections. 
In addition, designated visitor offices are available. 
The DSC also manages approximately 30 servers for core network services using primarily generic Intel-compatible servers running CENTOS. 
%Special staff support is offered for publications and graphics.
The lab hosts 5 international visiting Professors which trains many post-docs and has many federal, European and private foundations projects all in interdisciplinary medical imaging research.

\emph{ViCOROB} is a research institute specialized in computer vision and robotics at \ac{udg}.
The laboratories of ViCOROB are well-equipped with computers, servers, and specific software required for processing clinically data acquired. 
The Image Analysis Lab has recently been equipped with 2 high-performance servers (featuring 4 quad-core processors and 128 GB of RAM), a Totoku MS31i2 Diagnostic Displays system, and access to the use of CESCA facilities (the Supercomputing research center in Barcelona) which offers supercomputing shared-memory and distributed-memory machines suitable when dealing with such huge amount of data.
Furthermore, the fellow will have access to a wide range of scientific journals and books via institution authentication.

% {\color{red} \subsection{Competences, experience and complementary of the participating organisations and institutional commitment}
% \label{sec:competences}

% USA has been the number one research destination for young Europeans under every single international funding instrument. 
% This reflects upon the unlimited opportunities as well as the accumulated research experience and competence.
% As it is stated in the institutional web, there is probably no other large public university with a stronger culture of commitment to student success than \ac{fsu}.
% \ac{fsu} will provide the institutional, computational, high-quality research and human resources necessary to accomplish the objectives of this project.
% \ac{udg} has a successful tradition in acquiring funding from the European Commission, Marie-Curie projects, where 8 are Marie-Curie fellowships.
% The institutional embedding in the ViCOROB, the return phase allocation for the researcher, will encourage the potentialities of the proposed aims within this project, with a constant flow of international researchers, seminars and workshops, and a modern and competitive environment including last technologies for communications or cluster computer's room available to the applicant.
% Strengthening the links of both institutions in terms of application and training in research methods and cutting-edge technologies serves as bridge to connect and enrich them in terms of knowledge and culture and open new doors for international exchange programs and lasting collaborations.

% THIS IS NOT IN THE PROPOSAL ANYMORE APPARENTLY.

% }
\newpage

\begin{landscape}

  \begin{figure}[htbp]
    \begin{center}

      \begin{ganttchart}[
        y unit chart=.35cm,
        canvas/.append style={fill=none, draw=black!5, line width=.75pt},
        hgrid style/.style={draw=black!5, line width=.75pt},
        vgrid={*1{draw=black!5, line width=.75pt}},
        title/.style={draw=none, fill=none},
        title label font=\bfseries\footnotesize,
        title label node/.append style={below=7pt},
        include title in canvas=false,
        bar label font=\small\color{black!70},
        bar label node/.append style={left=2cm},
        bar/.append style={draw=none, fill=black!63},
        bar progress label font=\footnotesize\color{black!70},
        bar height=.7,
        group left shift=0,
        group right shift=0,
        group height=.5,
        group peaks tip position=0,
        group label node/.append style={left=.6cm},
        group progress label font=\footnotesize,
        group label font=\small\color{black},
        %group top shift=.6,
        milestone/.append style={fill=orange!70}
        ]{1}{36}
        \gantttitle[
        title label node/.append style={below left=7pt and -3pt}
        ]{Month:\quad1}{1}
        \gantttitlelist{2,...,36}{1} \\
        \ganttgroup[group/.append style={fill=Goldenrod!100}]{WP1}{1}{5}\\
        \ganttbar[bar/.append style={fill=Green!70}]{T1.1}{1}{3}\\
        \ganttbar[bar/.append style={fill=Green!70}]{T1.2}{1}{4}\\
        \ganttbar[bar/.append style={fill=Green!70}]{T1.3}{5}{5}\\
        \ganttbar[bar/.append style={fill=Green!70}]{T1.4}{5}{10}\\
        \ganttbar[bar/.append style={fill=Red!70}]{R1}{5}{5}\\
        \ganttgroup[group/.append style={fill=Goldenrod!100}]{WP2}{6}{12}\\
        \ganttbar[bar/.append style={fill=Green!70}]{T2.1}{6}{7}\\
        \ganttbar[bar/.append style={fill=Green!70}]{T2.2}{8}{9}\\
        \ganttbar[bar/.append style={fill=Green!70}]{T2.3}{10}{11}\\
        \ganttbar[bar/.append style={fill=Green!70}]{T2.4}{12}{12}\\
        \ganttbar[bar/.append style={fill=Red!70}]{R2}{12}{12}\\
        \ganttgroup[group/.append style={fill=Goldenrod!100}]{WP3}{13}{19}\\
        \ganttbar[bar/.append style={fill=Green!70}]{T3.1}{13}{14}\\
        \ganttbar[bar/.append style={fill=Green!70}]{T3.2}{15}{17}\\
        \ganttbar[bar/.append style={fill=Green!70}]{T3.3}{18}{19}\\
        \ganttbar[bar/.append style={fill=Red!70}]{R3}{19}{19}\\
        \ganttgroup[group/.append style={fill=Goldenrod!100}]{WP4}{20}{26}\\
        \ganttbar[bar/.append style={fill=Green!70}]{T4.1}{20}{23}\\
        \ganttbar[bar/.append style={fill=Green!70}]{T4.2}{24}{26}\\
        \ganttgroup[group/.append style={fill=Goldenrod!100}]{WP5}{27}{34}\\
        \ganttbar[bar/.append style={fill=Green!70}]{T5.1}{27}{28}\\
        \ganttbar[bar/.append style={fill=Green!70}]{T5.2}{29}{32}\\
        \ganttbar[bar/.append style={fill=Green!70}]{T5.3}{33}{34}\\
        \ganttbar[bar/.append style={fill=Red!70}]{R4}{34}{34}\\
        \ganttgroup[group/.append style={fill=Goldenrod!100}]{WP6}{35}{36}\\
        \ganttbar[bar/.append style={fill=Green!70}]{T6.1}{35}{35}\\
        \ganttbar[bar/.append style={fill=Green!70}]{T6.2}{35}{35}\\
        \ganttbar[bar/.append style={fill=Green!70}]{T6.3}{36}{36}\\
        \ganttgroup[group/.append style={fill=Goldenrod!100}]{WP7}{1}{36}\\
        % \ganttgroup[group/.append style={fill=Goldenrod!100}]{WP7}{1}{36}\\
        % \ganttbar[bar/.append style={fill=Green!70}]{T7.1}{1}{36}\\
        % \ganttbar[bar/.append style={fill=Green!70}]{T7.2}{6}{6}
        % \ganttbar[bar/.append style={fill=Green!70}]{}{12}{12}
        % \ganttbar[bar/.append style={fill=Green!70}]{}{18}{18}\\
        % \ganttbar[bar/.append style={fill=Green!70}]{T7.3}{1}{36}\\
        % \ganttbar[bar/.append style={fill=Green!70}]{T1.4}{5}{10}\\
        % \ganttgroup[group/.append style={fill=Goldenrod!100}]{WP8}{1}{36}\\
        \ganttdeliv[inline]{D1.1}{5}{5}
        \ganttdeliv[inline]{D1.2}{10}{10}
        \ganttdeliv[inline]{D2.1}{8}{8}
        \ganttdeliv[inline]{D2.2}{11}{11}
        \ganttdeliv[inline]{D2.3}{12}{12}
        \ganttdeliv[inline]{D2.4}{13}{13}
        \ganttdeliv[inline]{D3.1}{17}{17}
        \ganttdeliv[inline]{D3.2}{18}{18}
        \ganttdeliv[inline]{D3.3}{19}{19}
        \ganttdeliv[inline]{D4.1}{29}{29}
        \ganttdeliv[inline]{D4.2}{30}{30}
        \ganttdeliv[inline]{D5.1}{34}{34}
        \ganttdeliv[inline]{D5.2}{35}{35}
        \ganttdeliv[inline]{D6.1}{36}{36} \\
        \ganttmile[inline]{M1.1}{2}{2}
        \ganttmile[inline]{M1.2}{5}{5}
        \ganttmile[inline]{M2.1}{12}{12}
        \ganttmile[inline]{M3.1}{16}{16}
        \ganttmile[inline]{M3.2}{19}{19}
        \ganttmile[inline]{M4.1}{23}{23}
        \ganttmile[inline]{M4.2}{26}{26}
        \ganttmile[inline]{M5.1}{30}{30}
        \ganttmile[inline]{M5.2}{34}{34}
        \ganttmile[inline]{M6.1}{36}{36} \\
        \ganttcomm[inline]{C2}{5}{5}
        \ganttcomm[inline]{C1}{12}{12}
        \ganttcomm[inline]{C1}{19}{19}
        \ganttcomm[inline]{C1}{26}{26}
        \ganttcomm[inline]{C1}{34}{34}
        \ganttcomm[inline]{C3}{36}{36}
        % \ganttgroup[group/.append style={fill=yellow!70}]{WP1}{1}{5} \\
        % \ganttbar[bar/.append style={fill=OliveGreen!50}]{D1.1}{5}{5} \\
        % \ganttmilestone{M1.1}{10} \\
        % \ganttgroup[group/.append style={fill=yellow!70}]{WP2}{5}{12} \\
        % \ganttbar[bar/.append style={fill=OliveGreen!50}]{D2.1}{12}{12} \\
        % \ganttmilestone{M2.1}{8} \\
        % \ganttmilestone{M2.2}{8} \\
        % \ganttmilestone{M2.3}{13} \\
        % \ganttmilestone{M2.4}{12} \\
        % \ganttgroup[group/.append style={fill=yellow!70}]{WP3}{12}{19} \\
        % \ganttbar[bar/.append style={fill=OliveGreen!50}]{D3.1}{19}{19} \\
        % \ganttmilestone{M3.1}{19} \\
        % \ganttmilestone{M3.1}{19} \\
        % \ganttgroup[group/.append style={fill=yellow!70}]{WP4}{19}{26} \\
        % \ganttbar[bar/.append style={fill=OliveGreen!50}]{D4.1}{26}{26} \\
        % \ganttmilestone{M4.1}{26} \\        
        % \ganttgroup[group/.append style={fill=yellow!70}]{WP5}{26}{34} \\
        % \ganttbar[bar/.append style={fill=OliveGreen!50}]{D5.1}{34}{34} \\
        % \ganttmilestone{M5.1}{34} \\
        % \ganttgroup[group/.append style={fill=yellow!70}]{WP6}{34}{36} \\
        % \ganttbar[bar/.append style={fill=OliveGreen!50}]{D6.1}{36}{36} \\
        % \ganttmilestone{M6.1}{36} \\
        %\ganttgroup[group/.append style={fill=yellow!70}]{WP7}{1}{36} \\
        %\ganttgroup[group/.append style={fill=yellow!70}]{WP8}{1}{36} \\
      \end{ganttchart}

    \end{center}
    \caption{Gantt Chart of the proposal.}
  \end{figure}

\end{landscape}
\newpage