\section{IMPLEMENTATION}
\label{sec:implementation}

\subsection{Overall coherence and effectiveness of the work plan}

The proposal is separated into 7 work packages:

\begin{description}
\item[WP1: Data acquisition and dissemination (duration of 5 months)]


  \begin{description}
  \item[T1.1] Preparation of the multi-parametric \ac{mri} dataset
  \item[T1.1] Preparation of the multi-parametric \ac{mri} dataset
  \end{description}
\item[WP2: Pre-processing (duration of 7 months)]
\item[WP3: Segmentation (duration of 7 months)]
\item[WP4: Registration (duration of 7 months)]
\item[WP5: Detection and assessment (duration of 8 months)]
\item[WP6: \ac{pirads} evaluation (duration of 2 months)]
\item[WP7: Communication (duration of 36 months)]
\end{description}

\subsection{Appropriateness of the management structure and procedures, including quality management and risk management}

Develop your proposal according to the following lines:
\begin{itemize}
\item Project organisation and management structure, including the financial management strategy, as well as the progress monitoring mechanisms put in place;
\item Risks that might endanger reaching project objectives and the contingency plans to be put in place should risk occur.
\end{itemize}
The following could be also included in the Gantt Chart:
\begin{itemize}
\item Progress monitoring;
\item Risk management;
\item Intellectual Property Rights (IPR).
\end{itemize}

\newpage

\begin{landscape}

\begin{figure}[htbp]
\begin{center}

\begin{ganttchart}[
    canvas/.append style={fill=none, draw=black!5, line width=.75pt},
    hgrid style/.style={draw=black!5, line width=.75pt},
    vgrid={*1{draw=black!5, line width=.75pt}},
    title/.style={draw=none, fill=none},
    title label font=\bfseries\footnotesize,
    title label node/.append style={below=7pt},
    include title in canvas=false,
    bar label font=\small\color{black!70},
    bar label node/.append style={left=2cm},
    bar/.append style={draw=none, fill=black!63},
    bar progress label font=\footnotesize\color{black!70},
    group left shift=0,
    group right shift=0,
    group height=.5,
    group peaks tip position=0,
    group label node/.append style={left=.6cm},
    group progress label font=\bfseries\small
  ]{1}{24}
  \gantttitle[
    title label node/.append style={below left=7pt and -3pt}
  ]{Month:\quad1}{1}
  \gantttitlelist{2,...,24}{1} \\
  \ganttgroup{Work Package}{1}{10} \\
  \ganttgroup{Deliverable}{5}{15} \\
  \ganttgroup{Milestone}{5}{5} \\
  \ganttgroup{Secondment}{20}{23} \\
  \ganttgroup{Conference}{16}{16} \\
  \ganttgroup{Workshop}{17}{17} \\
  \ganttgroup{Seminar}{18}{18} \\
  \ganttgroup{Dessemination}{23}{24} \\
  \ganttgroup{Public engagement}{4}{5} \\
  \ganttgroup{Other}{7}{10}
\end{ganttchart}

\end{center}
\caption{Example Gantt Chart}
\end{figure}

\end{landscape}

\newpage

\subsection{Appropriateness of the institutional environment (infrastructure)}
\label{sec:institution}

Give a description of the legal entity/ies and its main tasks (per participant).
Explain why the fellowship has the maximum chance of a successful outcome.

NB: Each participant is described in Section 6. This specific information should not be repeated here.

\subsection{Competences, experience and complementarity of the participating organisations and institutional commitment}
\label{sec:competences}

Here describe how the fellowship will be beneficial for both the Experienced Researcher and host organisation(s).
\begin{itemize}
\item Commitment of beneficiary and partner organisations to the programme (for partner organisations, please see also section 6)
\end{itemize}

\paragraph{Partner organisations:}
The role of Partner organisations in MS/AC for secondments and their active contribution to the research and training activities should be described.
