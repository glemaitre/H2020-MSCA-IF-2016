\section{EXCELLENCE}
\label{sec:excellence}

Please note that the principles of the European Charter for Researchers and Code of Conduct for the Recruitment of Researchers promoting open recruitment and attractive working conditions are expected to be endorsed and applied by all beneficiaries in the Marie Sklodowska-Curie actions.

\subsection{Quality, innovative aspects and credibility of the research}
\label{sec:quality}

You should develop your proposal according to the following lines:
\begin{itemize}
\item Introduction, state-of-the-art, objectives and overview of the action
\item Research methodology and approach: highlight the type of research and innovation activities proposed
\item Originality and innovative aspects of the research programme: explain the contribution that the project is expected to make to advancements within the project field. Describe any novel concepts, approaches or methods that will be employed.
\end{itemize}
The text should emphasise how the high-quality, novel research is the most likely to open up the best career possibilities for the Experienced Researcher and new collaboration opportunities for the host organisation(s).

\subsection{Clarity and quality of transfer of knowledge/training for the development of the researcher in light of the research objectives}
\label{sec:transfer}

A two way transfer of knowledge should be described (please see Section 5.2 of this Guide):
\begin{itemize}
\item The text must show how the Experienced Researcher will gain new knowledge from the hosting organisation(s) during the fellowship through training.
\item These organisations may also benefit from the previous experience of the researcher. Outline the capacity for transferring the knowledge previously acquired by the researcher to the host organisation(s).
\end{itemize}

\subsection{Quality of the supervision and the hosting arrangements}
\label{sec:supervision}

Required sub-heading:
\subsubsection*{Qualifications and experience of the supervisor(s)}

Information regarding the supervisor(s) must include the level of experience on the research topic proposed and document its track record of work, including the main international collaborations. Information provided should include participation in projects, publications, patents and any other relevant results.
To avoid duplication, the role and profile of the supervisor(s) should only be listed in the "Capacity of the Participating Organisations" tables (see section 6 below).
The text must show that the Experienced Researcher should be well integrated within the hosting organisation(s) in order that all parties gain the maximum knowledge and skills from the fellowship.
The following section of the European Charter for Researchers refers specifically to career development:

\paragraph{Career development}
Employers and/or funders of researchers should draw up, preferably within the framework of their human resources management, a specific career development strategy for researchers at all stages of their career, regardless of their contractual situation, including for researchers on fixed-term contracts. It should include the availability of mentors involved in providing support and guidance for the personal and professional development of researchers, thus motivating them and contributing to reducing any insecurity in their professional future. All researchers should be made familiar with such provisions and arrangements.

\subsection{Capacity of the researcher to reach and re-enforce a position of professional maturity in research}
\label{sec:maturity}

Please keep in mind that the fellowships will be awarded to the most talented researchers as shown by their ideas and their track record, where it is a fair indicator given their level of experience.
