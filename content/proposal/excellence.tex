\section{EXCELLENCE}
\label{sec:excellence}

\subsection{Quality, innovative aspects and credibility of the research}
\label{sec:quality}

\subsubsection{Introduction}

In Europe, prostate cancer is reported to be the most frequently diagnosed cancer of men and thus one of the leading cause of death of cancer~\footcite{Ferlay2013}. 
Currently, addressing this issue is a major public debate, in which the implementation of appropriate screening methods and subsequent treatments is key.

In this regard, the \ac{erspc} is conducted to investigate the potential benefits of a population-based screening~\footcite{Schroder2015}. 
The screening consists of a \ac{psa} test and depending of the \ac{psa} level measured, an additional ``blind'' biopsy is carried out. 
Despite the fact of a significant reduction of the prostate cancer mortality, the screening strategy employed suffers of a high rate of over-diagnosis and over-treatment~\cite{Delpierre2013,Schroder2015}, due to the fact that prostate cancer growth is characterized by two main types of evolution: slow and fast.
The slow-growing tumours account for up to 85 \% of all cancers and stay confined to the prostate gland, while the fast-growing tumours rapidly develop and metastasise to other organs, significantly affecting the morbidity and mortality rate.
Furthermore, prostate cancer is more likely to develop in specific regions of the prostate: around 70-80 \% of prostate cancers originate in the \ac{pz}, whereas 10-20 \% in the \ac{cg}, but appear to be more aggressive and more likely to invade other organs. 
\textbf{Thus, together with cancer detection, the screening methods need to provide an estimate of the cancer aggressiveness to allow clinicians to act accordingly.}

In addition, the investigators of the \ac{erspc} have come to the conclusion that the use of \emph{``multi-parametric \ac{mri} and the development of new markers are the hope for the future''}.
That is why, \ac{cad} systems revolved around mono- and multi-parametric \ac{mri} are currently developed by the medical imaging community, and were recently reviewed by Lema\^itre~\emph{et al.}~\footcite{Lemaitre2015}.
The \ac{cad} systems developed are designed under the same architecture as depicted in Fig.\,\ref{fig:wkfcad}. 
The available \ac{mri} modalities during prostate exam are \ac{t2w}-\ac{mri}, \ac{dw}-\ac{mri}, \ac{dce}-\ac{mri}, and \ac{mrsi}. 
Additionally, \ac{adc} map is based on the computation of a coefficient derived from multiple \ac{dw}-\ac{mri}.
\textbf{Currently, no \ac{cad} system has been developed using all the available modalities and thus discarding their potential discriminating power to diagnose prostate cancer.}
\input{./content/proposal/figures/cad}
The closest attempts used three of these modalities (i.e., \ac{t2w}-\ac{mri}, \ac{dw}-\ac{mri}, \ac{dce}-\ac{mri}) and discarded \ac{mrsi}~\footcite{Litjens2014}\textsuperscript{,}\footcite{Viswanath2011}.
This latter, however, has been shown to be extremely helpful to grade cancer aggressiveness particularly in the \ac{cg}~\footcite{Vos2015}, which is the most challenging zone in terms of cancer detection.
\textbf{Furthermore, the current researches solely focus on the delineation of prostate cancers rather than on the cancer aggressiveness assessment.}

Therefore, the aim of this project is to design a \ac{cad} system able to both detect and assess prostate cancers using all currently available multi-parametric \ac{mri} modalities.
The architecture of our \ac{cad} system will imply the following investigations: 
\begin{enumerate}
\item Pre-processing to enhance the quality of \ac{mri} images (bias field correction, denoising, and normalisation),
\item Segmentation of prostate zones using multi-parametric \ac{mri} and deep-learning,
\item Registration of multi-parametric \ac{mri} using spline-based non-linear differmorphism,
\item Detection and assessment of prostate cancers using using multi-parametric \ac{mri} and deep-learning,
\item Identification of markers by inspection of the neural-network and transfer to classical machine learning approach.
\end{enumerate}
These methodologies will be extensively presented and argued in Sect.\,\ref{sec:methodologies}.

\subsubsection{Research methodologies}
\label{sec:methodologies}

\paragraph{Pre-processing}

\Ac{mri} images are corrupted by different phenomena: (i) bias field, (ii) noise, and (iii) inter-patient variations.
In this regard, particular attention to correct each of these drawbacks will be addressed.

\Ac{mri} images are affected by the inhomogeneity of the \ac{mri} field called bias field, resulting in a smooth variation of the intensities.
Although bias correction methods are commonly used to enhance brain \ac{mri} images~\footcite{Vovk2007}, only one \ac{cad} system for prostate has reported to use such pre-processing~\footcite{Viswanath2009}.
The same authors performed an empirical evaluation of the state-of-the-art methods~\footcite{viswanath2011empirical} concluding that N3 algorithm~\footcite{Sled1998} yields to better classification results than other methods~\footcite{Styner2000}\textsuperscript{,}\footcite{Cohen2000}.
Recently, Lin~\emph{et al.}~\footcite{Lin2011} proposed a method combining the N3 algorithm with the FCM algorithm~\footcite{Ahmed2002} which outperforms the original methods in terms of breast segmentation.
\textbf{Therefore, we will perform an empirical comparison of these state-of-the-art methods~\footcite{Sled1998}\textsuperscript{,}\footcite{Styner2000}\textsuperscript{,}\footcite{Cohen2000}\textsuperscript{,}\footcite{Lin2011}, by ensuring the benefits of the method of Lin~\emph{et al.} for our specific application.}

Apart of the bias field, \Ac{mri} images are also degraded due to a Rician noise.
Similarly to bias correction, only two \ac{cad} systems for prostate denoised images using wavelet-based techniques~\footcite{Mallat2008}\textsuperscript{,}\footcite{Pizurica2003}, offering a proper theoretical baseline for Rician corruption~\footcite{Nowak1999}.
Non-Local Means-based denoising techniques have never been used for \ac{mri} prostate images, but extensively and successfully for other \ac{mri} applications~\footcite{Manjon2008}.
\textbf{Thus, we will perform an empirical evaluation of the Non-Local Means-based techniques~\footcite{Manjon2012}\textsuperscript{,}\footcite{Coupe2011} and wavelet-based technique\footcite{Pizurica2003} to select the appropriate method to our application.}



\ac{mrsi} is a modality related to one dimensional signal, thus enhancing techniques differs from the one used in \ac{mri}.
The \ac{mrsi} spectra have to be corrected for several phenomena: phase correction, water and lipid residuals filtering, baseline correction, frequency alignment, and normalisation.
This set of enhancement techniques already have been investigated by Lema\^itre~\emph{et al.}~\footcite{Lemaitre2011}, focusing solely on the \ac{mrsi} modality for prostate cancer detection.




Bias correction using Styner and Ahmed.
Noise reduction apply non local mean specifically for MRI noise with Rician models which have been extensively used for brain imaging and all type of modalities. adaptive non local mean.
T2W normalisation, DCE normalisation, ADC normalisation.

\paragraph{Segmentation}

\paragraph{Registration}

\paragraph{Detection and assessment}

\subsection{Clarity and quality of transfer of knowledge/training for the development of the researcher in light of the research objectives}
\label{sec:transfer}


\subsection{Quality of the supervision and the hosting arrangements}
\label{sec:supervision}

\subsubsection*{Qualifications and experience of the supervisor(s)}


\paragraph{Career development}

\subsection{Capacity of the researcher to reach and re-enforce a position of professional maturity in research}
\label{sec:maturity}
