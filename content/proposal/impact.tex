\section{IMPACT}
\label{sec:impact}

\subsection{Enhancing the potential and future career prospects of the researcher}
\label{sec:enhancement}

There are many extremely important scientific, technological and socio-economical reasons for extensively pursuing both fundamental and applied research in this field.
According to the document ``2020 Vision for the European Research Area'', European research policy should be deeply rooted in European society.
This document establishes that European research should support knowledge advancement in fields of major public concern such as health, facilitating the free circulation of researchers, knowledge and technology.
The proposed research aims at developing the first comprehensive \ac{cad} system for prostate cancer that can be easily adapted for breast cancer detection and diagnosis.
It will constitute an unique tool for enhancing the radiologist's interpretation with quantitative measurements using intelligent \ac{cad} systems.
Based on these objective measurements, this novel \ac{cad} will have the potential to be easily integrated into the clinical work-flow, advancing \ac{cad} systems beyond the current ``computer-aided display''-stage and one-parametric imaging.
This research will enable the translation of basic and applied data mining and computer vision algorithms to many cancer research areas besides prostate. 
These will include the main lethal cancers such as breast and brain and will be also applicable in other disciplines such as neuroscience or study of neurodegenerative diseases since the engineering mechanisms are the same.
The system will enable the development of unique prostate cancer biomarkers unknown before through improved quantitative evaluation methods and data mining methods including deep-learning, that will pave the pathway to personalized management for future patients diagnosed with early-stage prostate cancer, leading to reduced cost in prostate cancer screening by ultimately reducing the number of unnecessary biopsies.

The methods and results included in this project present a wide range of multidisciplinary aspects which will maximise the impact on the researcher's activity on European society, impacting on:
\begin{itemize}[noitemsep]
\item Basic research: study of the registration, segmentation, and deep-learning algorithms and their properties.
Development of novel \ac{cad} methods to be used in large datasets and multi-parametric \ac{mri} images.
\item Strategic research: nowadays biomedical image processing is a relatively new discipline within signal processing with active challenging topics.
\item Applied research: applications to challenging problems with social relevance in Europe as aggressive and non-aggressive tumour detection for prostate cancer diagnosis.
\item Transfer of knowledge: development of a software with the results of the investigation to be used in real applications by companies, researchers and practitioners.
Dissemination of results in international journals and conferences.
\end{itemize}

Additionally, the students who will work on associated aspects of this proposal will acquire skills that they can take with them to public and private sector jobs within the European Union. 
Thus, the applicant would serve in a multifaceted role as supervisor, mentor, career advisor, and project coordinator, which would lead him into a position of professional maturity benefiting alumni, researchers, patients, physicians and industrial sector.
Therefore, the present project has impact in three fundamental sectors for European society: academics/education, health, and industry.

\subsection{Quality of the proposed measures to communicate the action activities to different target audiences}

\paragraph{Communication and public engagement strategy of the action}
The topic and potential results of this project are important for the general public.
G. Lema\^itre has experience in public engagement with research projects.
On several occasions he presented his work to a very broad scientific and non-scientific audience.
He contacted consumer groups to advocate for his research and draw the attention to early detection and diagnosis to a very deadly cancer among men.
Prostate cancer is extremely prevalent among African Americans in the US and talking to these under-represented and under-privileged groups will be extremely beneficial for the large-scale dissemination of the research results achieved in this project.
This project is planned to take advantage of the worldwide spreading possibilities of both languages (Spanish and English) and have a similar communication procedure through \ac{fsu} research news and \ac{fsu} channel and Girona press channel, media, scientific-spreading blogs, digital media (the applicant is an active user of Facebook and Twitter with a science-lover network of contacts), press and TV (Communication 1 (C1), see Gantt chart).
In addition, the applicant plans to participate in Open-Doors day activities to attract high-school students for interdisciplinary research and for the new direction scientific computing as well as publishing blog post to popularize the pattern recognition and medical aspects of his research.

\paragraph{Dissemination of the research results:}
The scientific community will clearly benefit from the PREDICATE project through the publication of results and methods in high impact and open access journals to accelerate dissemination, uptake of results and enable possible exploitation.
The project findings will be carefully analyzed, with the support of the host institution. Those not considered to be patented, exploited commercially and transferred, will be made available to the scientific community by participating in high-impact peer-reviewed journal articles and conference.
Research results will be shared to the research and clinician communities by reporting results in adequate support: pattern recognition, machine learning, and computer science aspects will be: (i) reported in high-impact journals such as Medical Image Analysis, IEEE Transactions on Medical Imaging, Pattern Recognition, Computer Vision and Image Understanding, or Pattern Recognition Letters and (ii) disseminated in recognized conference such as MICCAI, IEEE ISBI, SPIE Medical Imaging, IEEE EMBC, IEEE ICIP, or ICPR.
However, clinical findings linked to the use of the proposed \ac{cad} system, with the partnership of clinicians will be reported in journals such as Clinical Oncology, Radiology, European Radiology, Investigative Radiology, or Medical Physics, and in conferences such as the ECR, RSNA, or ISMRM.
For the publication of scientific papers it will be followed the institutional policy of \ac{udg} on ``Open Access to scientific information and communication'' approved on December 2011 and implemented starting with 2012.
Therefore, the scientific publications will be either included in a institutional repository DUGi (\texttt{http://dugi.udg.edu}) which is compliant with OpenAIRE (\texttt{https://www.openaire.eu/}) or published in scientific journals with supporting ``Open Access''.
Furthermore, pre-print version will be made available on arXiv and Zenodo platform\footnote{\texttt{https://zenodo.org/} - funded by CERN/OpenAIRE/EU H2020}.
Additionally, G. Lema\^itre  will create a project home page for the software (C2) and report findings, publications, and access to research data by using the Zenodo platform, as previously done for his scientific research\footnote{\texttt{http://i2cvb.github.io/}}.
Before publication, research data will be anonymized and deidentified.
Furthermore, coverage of essential results will be pursued on community internet sites for medical imaging professionals, such as MedicalPhysicsWeb and AuntMinnie.

% Sharing of data generated by this project is an essential part of the proposed outreach activities and will be carried out in presenting the research at international scientific meetings in radiology and medical imaging, high impact journals and books.
% An additional goal of this project, which has been outlined throughout this document, is the dissemination of the results by developing the \ac{cad} as a software tool with independent modules that will find applicability in many cancer research areas.
% He will re-visit and refine existing components of this software tool.
% G. Lema\^itre  will develop a versatile, modular, open-source toolbox of algorithms readily usable by radiologists which is platform independent, robust, fast and easy to use in routine clinical practice, test this toolbox and distribute it to the scientific community.
% He will create a project home page for the software (C2).
% Dissemination activities will be tailored to the various target groups and end users identified in the dissemination and exploitation plan, including: Scientific and clinical community, Researchers (medical imaging, nuclear imaging, molecular imaging, image processing, image science), Radiologists, Physicians in preventive clinical care, European and international professional organizations, Patients advocacy groups.
% As each end user group requires a tailored dissemination approach, Table~\ref{tab:dissemination} outlines how each target group will be reached.

% \begin{table}
%   \centering
%   \begin{tabular}{ p{3.5cm} p{9.5cm} }
%     \toprule
%     Target group & How they will be reached  \\
%     \midrule
%     Research community & The organisations to be approached include, but are not limited to, the European Society of Radiology (ESR), Radiological Society of North America (RSNA), International Society of Magnetic Resonance in Medicine (ISMRM), European Society of Molecular Imaging (ESMI), European Society of Medical Oncology (ESMO), American Society of Clinical Oncology (ASCO), World Molecular Imaging Society (WMIS), European Association of Nuclear Medicine (EANM), Society of Nuclear Medicine and Molecular Imaging (SNMMI), European Society of Breast Imaging (EUSOBI).
% The achievements and impact of the project will be disseminated by publishing in international, peer-reviewed journals (including Journal of Clinical Oncology, Radiology, European Radiology, European Journal of Nuclear Medicine and Molecular Imaging, Investigative Radiology, Medical Physics, Clinical Cancer Research, Journal of Clinical Oncology, Breast Cancer Research) and proceedings of scientific conferences, in  particular of the organisations listed above.
% Dissemination of tailored newsletters/fact sheets and promotional material, invitations to presentations or hands-on demonstrations at meetings and conferences and the ProDeepCAD website. Participation in International Day of Radiology campaign.
% Social Media: ResearchGate \& LinkedIn
%   \\
%     Clinical community & Project-related scientific presentations (oral, posters) at medical conferences
% Dissemination of tailored newsletters/Factsheets and promotional material, invitations to presentations or hands-on demonstrations at meetings and conferences and the International Day of Radiology campaign.
% Social Media: ResearchGate \& LinkedIn  \\
%     Industry & Meetings with industry representatives at large medical or technical conferences (e.g. European Congress of Radiology, EANM congress) and facilitated by the Office of Innovation at FSU.
% Tailored newsletters and promotional material, invitations to presentations at meetings and conferences and a dedicated industry section on the ProDeepCAD website.
% Social Media: ResearchGate \& LinkedIn
% \\
%     \bottomrule
%   \end{tabular}
%   \label{tab:dissemination}
% \end{table}

% This will allow future interested researchers to continue development of the toolbox and will preserve its utility to the community.
% He will be primarily responsible for software dissemination and support through direct training of experienced radiologists and residents, via conferences, radiological journals and online tutorials.
% In addition he will train radiologists how the proposed \ac{cad} system can be employed as a decision making mechanisms in prostate cancer.
% The fellow has also active accounts for spreading results in scientific social-networks as: Google Scholar and ResearchGate.

\paragraph{Exploitation of results and intellectual property:}

Host institutions, hereby \ac{fsu} and \ac{udg}, and the beneficiary will be required to bring in existing background knowledge and provide transparent access to other partners for the successful execution of the project.
Prior to the start of the project and signing the \emph{Grant Agreement}, the different parties will draw up a \emph{Partnership Agreement}.
\emph{Partnership Agreement} will specify the background brought into the project per partner and lay down rules for collaboration and matters regarding the usage of back- and foreground.
All project participants are expected to enter the \emph{Partnership Agreement} in order to ensure confidentiality and facilitate knowledge transfer.
% All Access Rights needed for the execution of this project are granted on a non-exclusive basis and are worldwide.
% No transfer costs shall be charged by any participant for the granting of Access Rights.
% Additionally, all consortium beneficiaries have unanimously agreed on the following principles that will govern the \ac{ip} terms in this project, and form the basis of the \ac{ip} terms in the consortium agreement: Solely generated \ac{ip} will be solely owned by the generating party with a first option to the network's private sector partners to a non-exclusive option to license foreground \ac{ip} for commercialisation.
% All intellectual property generated jointly during the course of the Project (``Joint Foreground \ac{ip}'') and worldwide patent rights and copyrights arising there from shall be jointly owned by the generating parties, and shall be managed in accordance with the terms of a Joint Ownership and Management Agreement (JOMA).
% The Project Coordinator is responsible for establishment of the Consortium Agreement and intellectual property rights \ac{ip} management throughout the project lifetime.

In about three years a direct knowledge utilization from the proposed research in form of a first prototype of a \ac{cad} for prostate cancer detection and diagnosis will emerge that could be tested in clinical routine.
This project offers a high probability that \ac{ip} of significant commercial value due to its novelty (by being the first comprehensive \ac{cad} for prostate cancer) and its translational research applications for example in breast cancer research.
The applicant in conjunction with the \ac{oitt} and the technology transfer office at \ac{fsu} will look for opportunities through patents and disclosures to transfer this \ac{ip} to the private sector.
All the necessary steps will be taken in order to protect, assess, transfer/license, any exploitable results arising from this project, in accordance with the rules of the 2020 Horizon Program for Research.
Protecting intellectual property generated within PREDICATE will be essential for securing the technological leadership of the participating institutions.
In order to ensure that valuable \ac{ip} is identified and appropriately protected at an early stage, all partners will review all scientific output from their respective WPs and seek \ac{ip} protection for the corresponding results if deemed appropriate.
To allow the proper exploitation of Foreground knowledge that will require the involvement of larger industrial parties, the partners will contact third-party companies.
If considered appropriate, additional collaborators will be approached by the project members during the course of the project.
The expertise of the participating academic partners, and the contacts that are already in place, it will be highly feasible to properly exploit the knowledge and the results generated within PREDICATE.
The exploitation of results will be based on the interaction with the \ac{oitt} at the \ac{udg} in order to define a strategy for a successful deployment of the developed \ac{cad} system (C3).


% We will exploit  the  project  results  by  generating  new  intellectual  property,  collaborating with SMEs as well as external industry representatives and developing a road-map towards widespread clinical application.

% Management of \ac{ip} rights will be elaborated in the Consortium Agreement according to the rules of the EC Model Contract.
% Foreground knowledge shall be owned by the contracting Party who has generated such knowledge.
% If the Parties jointly generate the Foreground, they shall have joint ownership on basis of the ratio of effort made to obtain the knowledge.
% The Parties concerned will seek to agree between them for obtaining and/or maintaining such shared rights on a case-by-case basis.
% To avoid merging of pre-existing know-how (Background) and know-how generated in this project (Foreground), pre-existing knowledge will be investigated before the start of the project, and drawn up as an annex to the Consortium Agreement.
% The SMEs interested in ProDeepCAD will be involved in all exploitation matters resulting from the outcome of this project.

