\section{IMPACT}
\label{sec:impact}

\subsection{Enhancing research- and innovation-related human resources, skills, and working conditions to realise the potential of individuals and to provide new career perspectives}
\label{sec:enhancement}

There are many extremely important  scientific, technological and socio-economical reasons for extensively pursuing  both fundamental and applied research in this field.  According to the document ‘2020 Vision for the European Research Area’, European research policy should be deeply rooted in European society. This document establishes that European research should support knowledge advancement in fields of major public concern such as health, facilitating the free circulation of researchers, knowledge and technology. The proposed research aims at developing  the first comprehensive CAD system for prostate cancer that can be easily adapted for breast cancer detection and diagnosis.  It will constitute an unique tool for enhancing the radiologist’s interpretation with quantitative measurements using intelligent CAD systems. Based on these objective measurements,  this novel CAD will have the potential to be easily integrated into the clinical work-flow, advancing CAD systems beyond the current “computer-aided display”-stage and one-parametric imaging. This research will enable the translation of  basic and applied data mining and computer vision algorithms to many cancer research areas besides prostate. These will include the main lethal cancers such as breast and brain. The system will enable the development of unique prostate cancer biomarkers unknown before that will pave the pathway to personalized management for future patients diagnosed with early-stage prostate t cancer.

In addition, it will improve early diagnosis and treatment of prostate cancer through improved quantitative evaluation methods  and data mining methods including deep learning that will lead to reduced cost in prostate cancer screening by ultimately reducing the number of unnecessary biopsies. Furthermore, the developed algorithmic tools will be relevant  for breast and brain cancer diagnosis as well, and will be also applicable in other disciplines such as neuroscience or study of neurodegenerative diseases since the engineering mechanisms are the same.

The methods and results included in this project present a wide range of multidisciplinary aspects which will maximise the impact on the researcher's activity on European society, impacting on:
-Basic research: study of the registration, segmentation, and deep learning  algorithms and their properties.  Development of novel CAD methods to be used in large datasets and multiparametric images.
-Strategic research: nowadays biomedical image processing is a relatively new discipline within signal processing with active challenging topics.
-Applied research: applications to challenging problems with social relevance in Europe  as aggressive and non-aggressive tumour detection for prostate cancer diagnosis.
-Transfer of knowledge: development of a software with the results of the investigation to be used in real applications by companies, researchers and practitioners.  Dissemination of results in international journals and conferences.

Additionally, the students who will work on associated aspects of this proposal will acquire skills that they can take with them to public and private sector jobs within the European Union. Thus, the applicant would serve in a multifaceted role as supervisor, mentor, career advisor and project coordinator, which would lead him into a position of professional maturity benefiting alumni, researchers, patients, physicians and industrial sector. Therefore, the present project has impact in three fundamental sectors for European society: academics/education, health and industry.

\subsection{Effectiveness of the proposed measures for communication and results dissemination}

Communication and public engagement strategy of the action: The topic and potential results of this project are important for the general public. Dr. Lemaitre has experience in public engagement with research projects. On several occasions he presented his work to a very broad scientific and non-scientific audience.  He contacted consumer groups to advocate for his research and draw the attention to early detection and diagnosis to a very deadly cancer among men.  Prostate cancer is extremely prevalent among African Americans in the US and talking to these under-represented and under-privileged groups will be extremely beneficial for the large-scale dissemination of the research results achieved in this project. This project is planned to take advantage of the worldwide spreading possibilities of both languages (Spanish and English) and have a similar communication procedure through FSU research news and FSU channel and Girona press channel, media, scientific-spreading blogs, digital media (the applicant is an active user of facebook and twitter with a science-lover network of contacts), press and TV (Communication 1 (C1), see Gantt chart). In addition, the applicant plans to participate in Open-Doors day activities to attract highschool students for interdisciplinary research and for the new direction scientific computing.
Dissemination of the research results: Sharing of data generated by this project is an essential part of the proposed outreach activities and will be carried out in presenting the research at international scientific meetings in radiology and medical imaging, high impact journals and books.
An additional goal of this project, which has been outlined throughout this document, is the dissemination of the results by developing the CAD as a software tool  with independent modules that will find applicability in many cancer research areas.  He will re-visit and refine existing components of this software tool.  Dr. Leamitre  will develop a versatile, modular, open-source toolbox of algorithms readily usable by radiologists which is platform independent, robust, fast and easy to use in routine clinical practice, test this toolbox and distribute it to the scientific community. He will create a project home page for the software (C2). This will allow future interested researchers   to continue development of the toolbox and will preserve its utility to the community.  He will  be primarily  responsible  for  software  dissemination  and  support  through  direct training of experienced radiologists and residents, via conferences, radiological journals and online tutorials. In addition he will train radiologists how the proposed CAD system can be employed as a decision making mechanisms in breast cancer.
The fellow has also active accounts for spreading results in scientific social-networks as: google-scholar, researchGate, scopus.

Exploitation of results and intellectual property:  In about three years a direct knowledge utilization from the proposed research in form of a first prototype of a CAD for prostate cancer detection and diagnosis  will emerge that could be  tested in clinical routine. This project offers a high probability that intellectual property (IP) of significant commercial value due to its novelty  ( by being the first comprehensive CAD for prostate cancer) and its translational research applications for example in breast cancer research. The applicant will look for opportunities through patents and disclosures to transfer this IP to the private sector. All the necessary steps will be taken in order to protect, assess, transfer/license, any exploitable results arising from this project, in accordance with the rules of the 2020 Horizon Program for Research. The exploitation of results will be based on: the manifested interest of Quiron group on SIPBA research activities; the help of living-lab salud Andalucia, a living lab agency operating in the regional south of Spain (Andalusia); and CESEAND, the andalusian node of the Enterprise Europe Network, which are already known contacts to the applicant (C3). (Please adapt for Giirona)

\paragraph{Dissemination, exploitation of results}
All researchers should ensure, in compliance with their contractual arrangements, that the results of their research are disseminated and exploited, e.g. communicated, transferred into other research settings or, if appropriate, commercialised. Senior researchers, in particular, are expected to take a lead in ensuring that research is fruitful and that results are either exploited commercially or made accessible to the public (or both) whenever the opportunity arises. 
